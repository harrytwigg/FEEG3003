\chapter{Introduction}

\noindent Teaching a computer to synthesize music using deep learning based methods has historically been a difficult task. Audio contains thousands of sound features every second in the time domain that are difficult to accurately teach to a neural network. Even if a network successfully learns how to interpret musical features, the output of models often sound fake or jarring to the listener due to inaccurate pitch and timbre representations and a lack of temporal context.

A subset of this topic is vocal sound synthesis using neural networks. This niche pertains to synthesizing singing and speech sounds using deep learning methods. This has historically proven more difficult than synthesizing instrumental based music due to the complexity of the human voice, with many different vocal modes and features that are difficult to learn through a neural network.

In this paper, the problem of difficulty surrounding synthesizing the human voice and singing is investigated. A variety of approaches have been proposed to solve this problem, with varying degrees of success, the methods and the results of which are discussed in this paper.

The goal of this paper is to compare the various approaches to vocal sound synthesis (singing), evaluating the merits and limitations of each approach. With the ultimate goal of building on the best of the existing methods further, demonstrating the applicability of the proposed approach.

Many potential applications would be oppened up if a deep learning model could be devised that could accurately learn, understand, and synthesize the fundamental features of music. These uses include many potentially artistic and business applications:

\begin{itemize}
    \item Rapidly synthesizing new vocal tracks for music production.
    \item Pitch transposing a piece of music eg transposing a piece of music down an octave, or up an octave.
    \item Changing room accoustics, eg if a piece of music was played in any echoic room, the model could be used to re-synthesize the same piece of music in an anechoic environment.
    \item Change the singing voice in a particular piece of music, in a similar fashon to deepfakes.
    \item Musical remixes of existing songs with different singers.
    \item Potentialy brand new forms of artistic expression, with a neural network perhaps being able to produce vocal features that are impossible to create naturally.
\end{itemize}

\section{Aim}

The aim of this project is to synthesize realistic human singing, using DDSP, with modifiable latent space parameters.

\section{Objectives}

\begin{enumerate}
    \item Build a functioning machine learning model that can synthesize human singing with modifiable latent space parameters (F0 and Z).
    \item Testing how accurate and adaptable the DDSP model is at synthesizing singing using several different inferencing tests, these validate that any created model behaves inline with the original DDSP model\cite{OriginalDDSP}, and would be useful in future applications. They are as follows:
    \begin{itemize}
        \item Almost identical resynthesis of the same song, using the same latent space parameters, and using the same artist as was used in the model's training dataset.
        \item Indepedent control of the latent parameters loudness and fundamental frequency (F0).
        \item Ensuring the words in model resynthesis are pronounced correctly (i.e. no babling or other audible errors).
        \item Ensuring the model can be trained in a reasonable amount of time, and inferred in at least real-time.
    \end{itemize}
    \item Derive constructive recommendations for future research based on past research and the results of this paper.
\end{enumerate}

\section{Methodology}

A critical review of existing literature on music sound synthesis (with a focus on vocal sound synthesis) was conducted. A standardised processs was developed to evaluate the quality of existing methods. This was necessary as it would be difficult to compare the results of different methods. Each of the approaches evaluated used different levels of abstraction and resolution of musical and audio data. Additionally they have different trade-offs in terms of accuracy and computational efficiency.

The standardised process is based on good principles in machine learning, along with accademic best practices. The technical requirements are as follows:

\subsection{Technical Requirements}

\begin{itemize}
    \item Overly time consuming methods should be penalised due to the limited time for the project.These can come in many forms eg. excessive training and computation time or extremely large datasets requirements, excessive hyperparameter tuning, or overly large networks
    \item Use of teacher forcing or operator involvement in any methods, as this will lead to biases in the model outputs and limits the scalability and ease of use of any derived models. Manually labelled data shall also be penalised similarly.
    \item Poor tonal quality in the output, eg. it is noticeable that the model was generated digitally as opposed to recorded. This could be caused by:
          \begin{itemize}
              \item Spectral leakage due to inaccuracies in fourier representations
              \item Poor oscillatory output representation that does not sound natural
          \end{itemize}
    \item To analyse the tonal quality Loudness and Fundamental Frequency should be compared using L1 - the sum of the absolute differences between the predicted and actual values when comparing output values to the training data. This is a good measure of the quality of the output.
    \item Modular systems shall be evaluated positively, due to the fact that their individual elements can be built on seperately, and the whole system acts less like a 'Black Box'.
    \item Any discarded information eg phase that has been discarded during encoding (eg. phase) that could be presented to the network shall also be penalised. It is hypothesized that this information could be used to improve the quality of the output.
    \item Model\cite{Attention} architectures that are specific to music and audio signal processing, as opposed to more general machine learning problems. It was believed that directing the model towards specific musical features (such as harmonics and pitch) would be beneficial, rather than generalising to the entire audio signal.
\end{itemize}

\subsection{Academic Requirements}

Well citided papers or those that are in scientific journals were looked upon favourablly, showing that other people have found the work to be valuable and more importantly, credible. It was also desired that any researched papers have open sourced code, and that the code is available for use. Without this the model cannot be easily built off of without building the codebase from the ground up, which would take considerable time. Older methods that have not been built further were evaluated negatively, as this suggests that experts in the field have judged the work to be of no further benefit and hence obscelete.

\subsection{DDSP Training}

After the intial research, DDSP was investigated further as it was the most promising method due to several factors discussed later in \nameref{section:DDSP}.

Two different datasets were created, one male voice artist and one female voice artist, to see how the DDSP architecture would handle male and female voices differently.

For each artist two different albums were picked of similar musical style to ensure consistency of vocal style across the entire dataset.

The albums were picked so that it only had one voice on the vocal track to avoid any problems of the model mixing voices, any songs with cover artists or different singers to the main voice were removed.

Each dataset was processed through a pre-trained model called Spleter\cite{Spleeter}. This pretrained model seperated the vocal track from the instrumentals for each song in the album.

The albums were picked so that it only had one voice on the vocal track to avoid any problems of the model mixing voices.

Each dataset was then trained using the DDSP library\cite{DDSPPip} and code adapted from a variation of DDSP designed for singing\cite{SingingDDSP}. Model hyperparameters were kept the same as in the Singing DDSP paper\cite{SingingDDSP} as the researchers had demonstrated thorough testing of which hyperparameters were the best.

\subsection{DDSP Inferencing}

Following training of both models, the models underwent several inferencing tests:

\begin{enumerate}
    \item The models were inferenced on the same dataset that was used for training to test the models acruaccy at predicting the training dataset.
    \item A pitch transposition was attempted. Vocal samples from the original datasret were transposed, up an and down an octave to see how the model would perform on unseen vocal ranges.
    \item Two different vocal soundtrack, one matching the vocal style of each dataset, was passed into the trained model. Each vocal track seperated similarly using Spleeter and had pitch, amplitude and timbre levels comparable to those of the original dataset. It was then seen if could recreate the vocal track in the style of the trained vocal artist. 
\end{enumerate}

Finally, in light of the experimental results, recommendations for future work are made.