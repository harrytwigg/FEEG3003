\chapter{An Investigation in Using DDSP to Learn and Synthesize Vocal Features}

The adaptation of DDSP to synthesize vocal features such as singing using the additional MFFC layer was chosen for further investigation.

All research was undertaken using Google Colab notebooks and cloud hardware, primarily NVIDIA Tesla V100 GPUs.

Two separate models were trained, one for a male voice (Chris Martin from Coldplay) and one for a female voice (Taylor Swift). This was done to see how the DDSP architecture would handle different vocies e.g. Alto or Tennor differently.

It was hypothesized that 2 albums would be the optimal size for any training dataset. Premilinary investigations on smaller datasets yielded significant overfitting and poor performance when any of the F0, amplitude latent features were modified during inference. Datasets of any greater size would be preferred. However, larger models would be slower to train.

\begin{figure}[!ht]
    \centering
    \caption{Songs and albums going into each training dataset}
\end{figure}

\section{Dataset Preparation}

\subsection{Source Separation}

Following the selection of the 2 albums for each artist and removal of any songs where additional vocal artists to the main vocalist featured were removed, all songs were passed through the pretrained Spleeter model\cite{SpleeterPip}\cite{SpleeterPip}. This pre-trained model is capable of carrying out source separation of instrumental and music tracks. This was essential as the training dataset must not contain any instrumentals.

The model outputted 2 seperate tracks for each song, one representing instrumentals and one vocals. The instrumental tracks were discarded. The remaining vocal tracks were then preprocessed.

Using Spleeter to split existing songs with instrumental components opened up the possability of using a greater number of songs, something the original singing DDPS paper's authors did not consider. Their largest single voice dataset had a preprocessed compressed size of ~70Mb, whereas the preprocessed datasets used in this paper had a size approximately 10 times that at ~700Mb, whilst still being based on a single vocal artist, style of music, and vocals only tracks. It was hoped using a larger training dataset would allow for less over-fitting and better generalisation.

\subsection{Pre-processing}

\begin{figure}
    \centering
    \includegraphics[width=0.6\textwidth]{research/dataset_preparation/PreprocessingSpecplot.png}
    \includegraphics[width=0.8\textwidth]{research/dataset_preparation/PreprocessingFeatures.png}
    \caption{Dataset Pre-processing: Spectrogram plot of a random 4 second sample from one of the datasets and its accomanying F0, F0 Confidence and Amplitude characteristics over time throughout the sample}
\end{figure}

Pre-processing of the datasets involved spliting the raw audio into smaller samples each 4 seconds long. The sample length was limited to 4 seconds to avoid capturing too much information in one spectrogram that would make learning using the convolutional neural network difficult.

For each sample F0 and confidence of F0 probability was inferred using CREPE\cite{CREPE}. Amplitude was computed statistically using the Librosa library\cite{LibrosaPip}. Latent Z information was available through the the passing of the raw audio. The 4 second samples and accomanying features were then stored as TFRecord files.

Each of the 2 datasets were preprocessed on Google Colab notebooks, this process took approximately 40 minutes for each dataset using a NVIDIA Tesla V100 GPU.

Finally, from each dataset a random 4 second clip was selected to prove successful pre-processing. Its its spectrogram was computed and plotted. Computed F0, F0 Confidence and Amplitude characteristics were also plotted for the selected clip. The underlying audio sample could also be played.

\section{Training}

Two seperate models were trained, one for each of the datasets.

A preprocessor was used that resampled the fundamental frequency and loudness, taking account the sample rate, frame rate, and number of timesteps. The number of timesteps was set at 1000 per 4 second clip, giving a spectral resoluiton of 4ms per timestep. This was deemed to be the best compromise between computational efficiency and accuracy.

An autoencoder encoder decoder setup was used. The encoder was based on a \acrfull{RNN}.

\section{Results}

To evaluate the performance of both models, several inferencing tests were conducted to see if the models had learned the latent features correctly.

\subsection{Recreation from the Training Dataset}

From each training dataset, a random frame was selected and passed through each model, loudness, F0 were kept constant. The results of the inferencing were then compared to the original training dataset.

Each model was able to successfully recrete original frames.

Timbral features were slightly distorted but the overall quality was good and it was easy to tell it was the original singer. Pitch estimation was highly accurate and in line with the original pitch. This can be partially attributed to the accruacy of the CREPE pitch detection model\cite{CREPE}.

A far bigger achievement was the re-synthesis of understandable words from the original frame. The original singing DDSP paper\cite{SingingDDSP} suffered a problem of stuttering when attempting resynthesis as their model was unable to recreate phonemes of the human voice accurately. It is possible that using far larger datasets has improved the quality of the resynthesis because the model has become more general in its ability to synthesize the human voice.

\begin{figure}
    \centering
    \includegraphics[width=0.8\textwidth]{research/results/TaylorSwift/InferredRecreation.png}
    \caption{(Taylor Swift) Original and resynthesized frames without latent modification}
\end{figure}

\subsection{F0 Pitch Transposition by a fixed octave}

A more advanced inferencing test was then undertaken, the fundamental frequency as determined by CREPE was transposed by fixed octaves (-2, -1, 0, 0.5, 1, 2) and the inferencing was re-preformed on the transposed frames.

At extreme transpositions, harmonics sometimes appeared to go silent. The resynthesis then appeared to sound like a whisper, coming almost entirely out of the filtered noise.

\begin{figure}
    \centering
    \includegraphics[width=\textwidth]{research/results/TaylorSwift/InferredTranspositions.png}
    \caption{(Taylor Swift) Inferred spectrogram frames at various octave transpositions realative to F0 at a certain timegrame in the original frame}
\end{figure}

\subsection{Fixing F0}

The final pitch related test was fixing F0 to the mean value of F0 throughout the frame to see how the models would perform under unnatural pitch conditions.

Both models were able to fix pitch to the mean value of F0 in the frame. This is clearly heard and can be seen visibily from the inferred spectrogram images where the harmonic components are at lines of constant frequency, unlike the orginial where they clearly vary.

This is a very good result mimicking what was founded in the Speech DDSP paper\cite{SpeechDDSP} (whose code wasn't publicly available). Further experimentations were done to vary the pitch to other amounts eg 100Hz 500Hz etc. to similar degrees of success, however the quality of results broke down at the extremes.

Additionally, words were still able to be synthesized and accurately heard, suggesting the model was able to learn the underlying phonemes of speech, or had at least learned how to pass them through the model.

Sadly, timbral quality was reduced when F0 was fixed, with the output sounding more robotic and the original timbre was lost. This is to be expected however as the original harmonic plus noise model was not designed with timbre transfer specifically in mind\cite{OriginalDDSP}.

\begin{figure}
    \centering
    \includegraphics[width=\textwidth]{research/results/TaylorSwift/FixedF0.png}
    \caption{(Taylor Swift) Training dataset and fixed F0 spectrogram frames}
\end{figure}

\begin{figure}
    \centering
    \includegraphics[width=\textwidth]{research/results/TaylorSwift/FixedF0Graphs.png}
    \caption{(Taylor Swift) Latent information on loudness and F0 over timesteps throughout the frame. The mean F0 was used to fix F0 throughout the frame}
\end{figure}