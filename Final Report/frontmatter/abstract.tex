\begin{center}
  \textsc{Abstract}
\end{center}
%
\noindent
%
In this report a novel machine learning technique called Differentiable Digital Signal Processing (DDSP) is investigated. This is a collection of machine learning models, differentiable versions of standard digital signal processing elements such as oscillators, noise filters, and a collection of other useful tools for learning how to decode, learn and subsquently synthesize new musical audio signals.

The DDSP model is applied to a set of vocal samples of a single artists voice, with the model extracting pitch, amplitude and timbre information from the vocal samples. The model is then trained to learn how to recreate the vocal samples from the extracted information. Different hyperparmeters and configurations of the model are then tested to see which one is best suited to learning the key vocal features.

The interpetable parameters are then manipuated to create new unheard samples from the learned model. Finally, the best trained model is applied to a different artist's vocal samples, and the second artist's song is made to be sound as if it was sung by the first artist.