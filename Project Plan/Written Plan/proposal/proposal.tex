\chapter{Proposal}

\label{ch:proposal}

\section{Background}

Machine Learning is rapidly transforming audio processing, yielding wide ranging
applications in business and for artists, amongst others. One particularly interesting
area is the use of deep based learning methods for music sound synthesis.

Machine Learning methods can already take musical or other sound data input, process
it, then provide a variety of potentially useful features, these include:
\begin{itemize}
    \item Timbre transfer eg. changing a melody from being in one instrument to
    another
    \item Pitch transfer eg. transposing an instrument up an octave
    \item Changing the room accoustics
    \item Generation of music and speach either randomly or from a given input
\end{itemize}

\section{Aim}

It is foreseeable that some machine learning sound synthesis methods have potential
to be built on further, whilst others may be deemed obsolete. This project aims
to compare these methods, evaluating the advantages and disadvantages of how audio
signals are encoded and presented to them. In an optimistic scenario, new methods
of audio signal encoding and deep learning methods for sound synthesis shall be
researched and proposed.

\section{Objectives}

\begin{itemize}
    \item Evaluate existing methods of audio signal encoding with a focus on the
    following machine learning areas and the way data is presented to the network
    (but not limited to):
    \begin{itemize}
        \item Frequency based Fourier coefficient methods eg. Tacotron, GANSynth
        \item Autoregressive whole single sample methods eg. Wavenet, SampleRNN
        \item DDSP a novel modular approach to sound synthesis
    \end{itemize}
    \item Investigate potentially innovative new methods of sound synthesis and
    encoding, building on existing researched work, this could include:
    \begin{itemize}
        \item The use of a new encoding method or formatting of the input signal
        \item Building on top of existing sound Synthesis Models, potentially making
        use of alternative ways of encoding and presenting data to the deep network.
        \item Finding a way of incorporating lost data eg. phase into the encoded
        data that is presented to the model.
    \end{itemize}
\end{itemize}

\section{Evaluation Methods}

It shall be necessary for the project to set out a set of criteria for carrying
out literary reviews on existing Learning Based Sound Synthesis Models. This must
be done to decide if any method has potential to be built on further.

\begin{itemize}
    \item Overly complex methods should be penalised. These can come in many forms
    eg. excessive training and computation time, extremely large datasets.
    \item Use of teacher forcing in any methods, as this will lead to biases in
    the model outputs.
    \item Poor tonal quality in the output, eg. it is noticeable that the model
    was generated digitally as opposed to recorded. This could be caused by
    \begin{itemize}
        \item Spectral leakage due to inaccuracies in fourier representations
        \item Poor oscillatory output representation that does not sound natural
    \end{itemize}
    \item Modular systems shall be evaluated positively, due to the fact that their
    individual elements can be built on seperately, and the whole system acts less
    like a 'Black Box'.
    \item Any discarded information eg phase that has been discarded during encoding
    (eg. phase) that could be presented to the network.
\end{itemize}